\documentclass[a4paper,10pt]{article}
\usepackage[margin=2cm]{geometry}
\usepackage[utf8]{inputenc}



\usepackage[round]{natbib}
\renewcommand{\bibname}{References}
\bibliographystyle{abbrvnat}

\usepackage{amsmath, amsfonts}
\usepackage{graphicx}
\usepackage{subfig}
\usepackage{float}
\usepackage{hyperref}
\hypersetup{colorlinks=true, urlcolor=blue, citecolor=magenta}


\makeatletter
\@addtoreset{exercise}{section}
\makeatother  
\newcounter{exercise}
\newtheorem{exercise}{Exercise}
\renewcommand{\theexercise}{\arabic{exercise}}
\newsavebox{\mybox}
\newenvironment{note}
{
\begin{center}
\begin{lrbox}{\mybox}
\begin{minipage}{42em}}
{\end{minipage}
\end{lrbox}\fbox{\usebox{\mybox}}
\end{center}}



\usepackage{amsmath, amsfonts, amsthm} 
\newcommand{\R}{\mathbb{R}}
\newcommand{\Hil}{\mathbb{H}}
\newcommand{\trnsp}{\mathsf{T}}
\newcommand{\N}{\mathcal{N}}
\newcommand{\Prob}[2]{\mathsf{P}_{#1}\left[#2\right]}


\title{Highlights: Neural Tangent Kernel}
\author{Antônio H. Ribeiro, Dave Zachariah, Per Mattsson}
\date{}

\begin{document}

\maketitle

Notes about the paper \textit{Neural Tangent Kernel: Convergence and Generalization in Neural Networks}~\citep{jacot_neural_2018}.


\section{Setup}

Let $f(\cdot; \theta): \R^d  \rightarrow \R$ be an almost everywhere diferentiable function  parametrized by $\theta\in \R^p$. Assume given a training dataset $\{(x_i, y_i)\}_{i-1}^n$ of input and outputs. We define the cost function
\begin{equation}
    V(\theta) = \frac{1}{2n} \sum_{i = 1}^n (f(x_i; \theta) - y_i)^2.
\end{equation}
Taking the derivative of $V$ we obtain,
\begin{equation}
    \nabla_\theta V(\theta) = \frac{1}{n} \sum_{i = 1}^n (f(x_i; \theta) - y_i) \nabla_\theta f(x_i; \theta).
\end{equation}

Now assume the parameter $\theta$ is estimated using gradient flow, let $\theta_t$ be the parameters estimated at the instant $t$. Then,
\begin{equation}
     \frac{d \theta_t}{d t}= - \eta \nabla_\theta V(\theta),
\end{equation}
and the chain rule yields
\begin{equation}
    \frac{d f(z, \theta_t)}{d t} = \eta \nabla_\theta V(\theta)^\trnsp \nabla_\theta f(z; \theta) =  \frac{\eta}{n} \sum_{i = 1}^n  (f(x_i; \theta) - y_i) \left( \nabla_\theta f(x_i; \theta)^\trnsp \nabla_\theta f(z; \theta)\right).
\end{equation}
We define the Neural Tangent Kernel, $K(\cdot, \cdot; \theta): \R^d \times \R^d  \rightarrow \R$
\begin{equation}
    K(x, z; \theta_t) = \nabla_\theta f(x; \theta)^\trnsp \nabla_\theta f(z; \theta).
\end{equation}
This is the kernel associated with the feature map $x \mapsto \nabla_\theta f(x,; \theta)$.  It then follows that
\begin{equation}
  \label{eq:training_ode}
    \frac{d f(z, \theta_t)}{d t} = \frac{\eta}{n} \sum_{i = 1}^n  (f(x_i; \theta) - y_i) K(x_i, z; \theta).
\end{equation}


\section{Model}

Here they consider $f(x; \theta) = \tilde\alpha^{(\ell)} (x, \theta)$ is the output of a neural network with $L$ layers. that could be defined recursively as:

\begin{equation}
\begin{aligned}
\alpha^{(0)}(x ; \theta) &=x \\
\tilde{\alpha}^{(\ell+1)}(x ; \theta) &=\frac{1}{\sqrt{n_{\ell}}} W^{(\ell)} \alpha^{(\ell)}(x ; \theta)+\beta b^{(\ell)} \\
\alpha^{(\ell)}(x ; \theta) &=\sigma\left(\tilde{\alpha}^{(\ell)}(x ; \theta)\right)
\end{aligned}
\end{equation}
where the nonlinearity $\sigma$ is applied entrywise and $\beta$ is a scaling factor. Here $\theta = ( W^{(1)},   b^{(1)},  W^{(2)},   b^{(2)}, \cdots,  W^{(L)},   b^{(L)})$. At initialization, each entry of $ W^{\ell}$  or $b^{\ell}$ is sampled from i.i.d Gausians $\mathcal{N}(0, 1)$. Hence, $\theta_0$ is a random variable. 



\section{Challenges}

Now there are two elements that make the model hard to deal with using traditional tools.
\begin{enumerate}
\item \textbf{The kernel is stochastic.} For the models studied $\theta_0$  is a random variable.
  Hence $K(\cdot, \cdot; \theta_0)$ is  not deterministic.
\item \textbf{The kernel is parametrized.} The kernel depends on a parameter $\theta$ that varies which is itself being updated during training.  Hence $K(\cdot, \cdot; \theta)$  is a kernel that evolves with the training. This reflects making the Eq.~\ref{eq:training_ode} not linear.
\end{enumerate}

\noindent
\textbf{Solutions} that are proposed in the paper:
\begin{enumerate}
\item  (Theorem 1) In probability, $K(x, y; \theta_0) \rightarrow K_0(x, y)$, i.e. where $K_0$ is a deterministic kernel.
\item (Theorem 2)  Uniformly on t, $K(x, y; \theta_t) \rightarrow K_0(x, y)$ for all $t \in [0, T]$.
\end{enumerate}

\section{Relation with other models}
We have presented another linear model approximation before, the one that assumed that $\theta  = \theta_0 + \beta$ and the number of parameters is so large that training effectively only changes the parameter by a small amount. Them $\beta$ is small and:
\begin{equation*}
f(z; \theta) \approx  f(x; \theta_0) + \nabla_\theta f(x; \theta_0)^\trnsp\beta.
\end{equation*}
In the case,  $f(z; \theta_0)$ this would come down to the map: $x \mapsto  \nabla_\theta  f(x; \theta_0)$ followed by the estimation of $\beta$ using a linear parameter. Notice that this nonlinear map coincides with the nonlinear map we are considering here.





\begin{thebibliography}{2}
\providecommand{\natexlab}[1]{#1}
\providecommand{\url}[1]{\texttt{#1}}
\expandafter\ifx\csname urlstyle\endcsname\relax
  \providecommand{\doi}[1]{doi: #1}\else
  \providecommand{\doi}{doi: \begingroup \urlstyle{rm}\Url}\fi

\bibitem[Belkin et~al.(2019)Belkin, Hsu, Ma, and
  Mandal]{belkin_reconciling_2019}
M.~Belkin, D.~Hsu, S.~Ma, and S.~Mandal.
\newblock Reconciling modern machine-learning practice and the classical
  bias\textendash variance trade-off.
\newblock \emph{Proceedings of the National Academy of Sciences}, 116\penalty0
  (32):\penalty0 15849--15854, Aug. 2019.
\newblock ISSN 0027-8424, 1091-6490.
\newblock \doi{10.1073/pnas.1903070116}.


\bibitem[Jacot et~al.(2018)]{jacot_neural_2018}
A. Jacot, F.~Gabriel, C.~Hongler.
\newblock {{Neural}} {{Tangent}} {{Kernel}}: {{Convergence}} and {{Generalization}} in {{Neural}} {{Networks}}
\newblock In \emph{Advances in {{Neural Information Processing Systems}} 32},
  2018.

\bibitem[Rahimi and Recht(2008)]{rahimi_random_2008}
A.~Rahimi and B.~Recht.
\newblock Random {{Features}} for {{Large}}-{{Scale Kernel Machines}}.
\newblock In \emph{Advances in {{Neural Information Processing Systems}} 20},
  pages 1177--1184. 2008.

  
\end{thebibliography}

\end{document}
